\documentclass[11pt,letterpaper]{article}
% Shared preamble for CHEM 1066 lab reports.
% This file loads the common styling and establishes course-wide defaults.
\usepackage{styles/chem1066-report}

% Set defaults that apply to every report; individual `.tex` files can override.
\SetChemReportCourse{CHEM 1066}
\SetChemReportInstitution{University of Minnesota -- Twin Cities}
\SetChemReportSemester{Fall 2025}
\SetChemReportAuthor{[Author]}
\SetChemReportHeader{Lab Report}
\SetChemReportPartners{[Lab Partners]}
\SetChemReportCoverTitle{Type Your Title Here:}
\SetChemReportCoverSubtitle{Concise and Informative Subtitle}
\SetChemReportPreparedOn{\today}

% Theme palette (choose exactly one)
\ChemReportUsePaletteDefault
% \ChemReportUsePaletteForestGold
% \ChemReportUsePaletteCoolTwilight


% Report-specific metadata
\SetChemReportHeader{Lab Report 1}
\SetChemReportCoverTitle{Freezing-Point Depression and Calorimetry of KCl:}
\SetChemReportCoverSubtitle{Determining the Van't Hoff Factor, Enthalpy of Dissolution, and Effectiveness as a Deicer}
\SetChemReportPartners{[Lab Partners]}
\SetChemReportPreparedOn{\today}

\begin{document}
\MakeChemReportCoverPage

% Abstract
\begin{center}
\section*{\centering Abstract}
\begin{adjustwidth}{1.5cm}{1.5cm}
    \textit{This experiment measured the van’t Hoff factor and enthalpy of dissolution for potassium chloride (KCl) to evaluate its potential as a deicer. The freezing‐point depression of aqueous KCl solutions was recorded to determine how solute concentration affects freezing temperature, and calorimetry was used to measure heat exchange during dissolution. The van ’t Hoff factor was found to be 1.90, indicating nearly complete ionic dissociation, while the enthalpy of dissolution was calculated as $1.4\;\text{kJ·mol}^{-1}$, suggesting a slightly exothermic process. Overall, these findings suggest that although KCl dissociates effectively to lower the freezing point of water, its small heat release upon dissolution makes it only moderately effective as a deicer compared to salts with strongly exothermic behavior.}
\end{adjustwidth}
\end{center}

\Needspace{3\baselineskip}
\section*{Introduction}

Each year over $118{,}100$ vehicular casualties occur as a result of snowy, slushy, or icy pavement and, consequently, United States state and local agencies invest billions of dollars to mitigate this issue, with a significant portion of the budget allocated to the dispersion of rock salt.\cite{ref1} Naturally, variations of rock salt are evaluated on several criteria—namely, their efficacy as deicers, cost, and environmental impact. These criteria, particularly deicer efficacy, can be analyzed using established chemical methods. For example, freezing‐point depression measurements can determine how effectively a salt lowers the freezing point of water, while calorimetry can reveal whether its dissolution is endothermic or exothermic—an important factor in how rapidly it can melt ice.

Previous research has shown that salts such as NaCl, KCl, and CaCl$_2$ vary significantly in both their thermodynamic and environmental properties. Calcium chloride, for instance, dissolves exothermically, releasing heat that accelerates melting, whereas potassium chloride dissolves endothermically, absorbing heat and thus melting ice more slowly.\cite{ref2} Understanding these energetic and colligative differences allows for more informed choices in deicer selection, balancing efficiency with environmental sustainability.

The objective of this experiment was to determine the van’t Hoff factor ($i$) and enthalpy of dissolution ($\Delta H$) of potassium chloride (KCl) in water to evaluate its effectiveness as a deicing compound. By examining how KCl affects the freezing point of water and the heat exchange involved in its dissolution, this work aims to connect measurable thermodynamic properties with the practical performance of KCl as an alternative to traditional road salts.

\Needspace{3\baselineskip}
\section*{Experimental}

\subsection*{Determination of the Freezing Point Depression}

A slurry of ice, water, and rock salt was prepared in a $250\;\text{mL}$ beaker until the temperature, measured with a Vernier temperature probe, was below $-10\degree\text{C}$. Four KCl solutions were made by measuring $1.00\;\text{g}$ of KCl on a scale and dissolving it into $8.0\;\text{mL}$, $12.0\;\text{mL}$, $16.0\;\text{mL}$, and $20.0\;\text{mL}$ of deionized water, respectively, measured using a $10\;\text{mL}$ graduated cylinder. Each solution was placed in a separate $15\;\text{mL}$ beaker.

Each beaker was placed so that its contents were below water level in the ice–salt bath while continuously monitoring the temperature. When ice crystals were first observed, the temperature was recorded as the freezing point of that solution. The freezing point depression was calculated using

\begin{equation}
\Delta T_f = |T_0 - T|
\label{eq:deltaTf}
\end{equation}

where $T_0$ is the freezing point of pure water and $T$ is the observed freezing point of the KCl solution.  
To determine the van’t Hoff factor ($i$), the freezing‐point depression was related to the solution molality ($m$) according to

\begin{equation}
\Delta T_f = iK_f m
\label{eq:vantHoff}
\end{equation}

where $K_f$ is the molal freezing‐point depression constant for water ($1.86\;\degree\text{C·kg·mol}^{-1}$). The slope of a plot of $\Delta T_f$ versus $m$ was used to obtain $iK_f$, from which $i$ was calculated in the Results section.

\subsection*{Determination of Calorimeter Heat Capacity}

A Styrofoam cup calorimeter was constructed with a cardboard lid sealed with duct tape and a hole to insert the temperature probe. Two $100\;\text{mL}$ beakers were filled with $40\;\text{mL}$ of deionized water each. One was cooled to approximately $10\degree\text{C}$ in an ice bath and the other heated to approximately $80\degree\text{C}$ on a hot plate. The cold water was transferred to the calorimeter, followed by the hot water with both temperatures measured and recorded immediately before mixing. The lid was immediately replaced, and the temperature was recorded until stable. The process was repeated for three trials.

The heat capacity of the calorimeter, $C_{\text{cal}}$, was determined using

\begin{equation}
C_{\text{cal}} = -\frac{mC_p(\Delta T_{\text{hot}}+\Delta T_{\text{cold}})}{\Delta T_{\text{cold}}}
\end{equation}

where $m$ is the mass of each water sample, $C_p$ is the specific heat of water, $\Delta T_{\text{hot}}$ is the temperature change of the hot water, and $\Delta T_{\text{cold}}$ is the temperature change of the cold water.

\subsection*{Determination of the Enthalpy of Dissolution of KCl}

After determining $C_{\text{cal}}$, $20\;\text{mL}$ of deionized water at room temperature was poured into the calorimeter. The initial temperature was recorded. Then, $1.00\;\text{g}$ of KCl was added, the lid was placed on, and the mixture was stirred until all solid dissolved. The final equilibrium temperature was recorded. This procedure was repeated for three trials. 

\Needspace{3\baselineskip}
\section*{Results}

\subsection*{Freezing Point Depression}

These data were plotted to illustrate the relationship between the freezing‐point depression and molality. The best‐fit line was constrained through the origin, as expected for colligative properties.

\Needspace{12\baselineskip}
\vspace{-0.8em}

\pgfmathsetmacro{\slope}{
 (0.675*2.1 + 0.858*3.8 + 1.125*4.3 + 1.722*5.6) /
 (0.675^2   + 0.858^2   + 1.125^2   + 1.722^2)
}

\begin{figure}[h]
\centering
\begin{tikzpicture}
\begin{axis}[
    width=0.75\textwidth,
    xmin=0, xmax=1.9,
    ymin=0, ymax=6.2,
    xlabel={Molality (mol/kg)},
    ylabel={$\Delta T_f$ (°C)},
    tick align=outside,
    grid=both,
    grid style={line width=.3pt, draw=gray!25},
    major grid style={line width=.5pt, draw=gray!50},
    legend style={draw=black, fill=white, font=\small, fill opacity=0.8},
    legend cell align={left},
    legend pos=south east
]
\addplot[
    only marks,
    mark=*,
    mark size=3pt,
    draw=\ChemReportSecondaryColor,
    fill=\ChemReportSecondaryColor!60!white
] coordinates {
    (0.675,2.1) (0.858,3.8) (1.125,4.3) (1.722,5.6)
};
\addlegendentry{Data}
\addplot[thick, domain=0:1.9, draw=\ChemReportPrimaryColor] {\slope*x};
\addlegendentry{Best Linear Fit Line}
\node[anchor=west, fill=white, inner sep=2pt] 
    at (axis cs:0.25,5.6)
    {$\Delta T_f = \pgfmathprintnumber[fixed,precision=3]{\slope}\,m$};
\end{axis}
\end{tikzpicture}
\vspace{-0.8em}
\caption{Freezing‐point depression versus molality with best‐fit line through origin.}
\label{fig:freezingpoint}
\end{figure}

\newpage

From the slope of the line in Figure~1 and using Equation~\ref{eq:vantHoff}, $iK_f = \pgfmathprintnumber[fixed,precision=3]{\slope}$, giving a calculated van’t Hoff factor of approximately

\[
i = 1.90.
\]
\subsection*{Calorimeter Calibration}

\begin{table}[h]
\centering
\caption{Calorimeter calibration temperature data}
\label{tab:calibrationdata}
\begin{tabular}{ccc}
\hline
$T_{\text{hot, initial}}$ (°C) & $T_{\text{cold, initial}}$ (°C) & $T_{\text{final}}$ (°C) \\
\hline
92.0 & 10.6 & 50.1 \\
81.6 & 8.1  & 43.8 \\
82.8 & 11.1 & 45.7 \\
\hline
\end{tabular}
\end{table}

Using this data and the equation

\begin{equation}
C_{\text{cal}} = -\frac{mC_p(\Delta T_{\text{hot}}+\Delta T_{\text{cold}})}{\Delta T_{\text{cold}}}
\label{eq:calorimeter}
\end{equation}

\noindent the heat capacities of the calorimeter for each trial were obtained as shown in Table~\ref{tab:calheat}.

\begin{table}[h]
\centering
\caption{Calculated calorimeter heat capacities}
\label{tab:calheat}
\begin{tabular}{lc}
\hline
& $C_{\text{cal}}$ (J/°C) \\
\hline
& 10.17 \\
& 9.85 \\
& 12.10 \\
\hline
& \textbf{Average: 10.71} \\
\hline
\end{tabular}
\end{table}

\subsection*{Enthalpy of Dissolution of KCl}

The measured heat changes during the dissolution of KCl are presented in Table~\ref{tab:dissolution}. These values were obtained using the calibrated calorimeter.

\begin{table}[h]
\centering
\caption{Reaction heat values for KCl dissolution}
\label{tab:dissolution}
\begin{tabular}{lc}
\hline
& $q_{\text{rxn}}$ (J) \\
\hline
& 175.2 \\
& 1532 \\
& 72.3 \\
\hline
& \textbf{Average: 593.4} \\
\hline
\end{tabular}
\end{table}

The heat of reaction and molar enthalpy were determined using the following relationships:

\begin{equation}
q_{\text{rxn}} = -C_{\text{cal}}\Delta T
\label{eq:rxn}
\end{equation}
\begin{equation}
\Delta H = \frac{q_{\text{rxn}}}{n_{\text{KCl}}}
\label{eq:enthalpy}
\end{equation}

Using the average heat of reaction and the number of moles of KCl dissolved, the calculated enthalpy of dissolution was found to be

\[
\Delta H_{\text{sol}} = 1.4\;\text{kJ/mol}.
\]



\Needspace{3\baselineskip}
\section*{Discussion}

The objective of this experiment was to determine the van’t Hoff factor ($i$) and enthalpy of dissolution ($\Delta H$) for potassium chloride (KCl) in water, in order to evaluate its effectiveness as a deicing compound. These quantities reveal both the ionic behavior of KCl in aqueous solution and the thermodynamic nature of its dissolution process.

As shown in Figure~\ref{fig:freezingpoint}, the freezing point depression ($\Delta T_f$) increased linearly with molality ($m$), consistent with colligative property theory. The slope of the best-fit line was $3.53\;\degree\text{C·kg·mol}^{-1}$. Substituting this value into the van’t Hoff relationship, {Equation~\ref{eq:vantHoff}}, with $K_f = 1.86\;\degree\text{C·kg·mol}^{-1}$ for water, yielded a van’t Hoff factor of $i = 1.90$. The theoretical value for KCl is 2, corresponding to dissociation into one $K^+$ and one $Cl^-$ ion.\cite{ref3} The small 5\% deviation suggests that the dissociation of KCl in water was nearly complete but slightly reduced by ion pairing or non-ideal electrostatic interactions between ions in solution. Minor temperature measurement uncertainty during freezing point observation or limited precision in the slope determination could also contribute to this deviation.

From the calorimetry portion, the average calorimeter heat capacity ($C_{\text{cal}}$) was calculated using {Equation~\ref{eq:calorimeter}} to be $10.71\;\text{J·°C}^{-1}$, as shown in Table~\ref{tab:calheat}. Using this value in {Equation~\ref{eq:rxn}} and the temperature changes measured during dissolution, the average heat of reaction ($q_{\text{rxn}}$) was determined to be $593.4\;\text{J}$. Applying {Equation~\ref{eq:enthalpy}} gave a dissolution enthalpy of $\Delta H_{\text{sol}} = 1.4\;\text{kJ·mol}^{-1}$. The positive sign indicates that the process was endothermic, meaning heat was absorbed from the surroundings as KCl dissolved. This observation aligns qualitatively with the known thermodynamic behavior of KCl dissolution, which is endothermic because lattice energy (the energy required to separate $K^+$ and $Cl^-$ ions) slightly exceeds the hydration energy released when these ions become solvated by water molecules.

The measured value of $1.4\;\text{kJ·mol}^{-1}$ is significantly smaller than the accepted literature value of approximately $+17\;\text{kJ·mol}^{-1}$.\cite{ref4} This difference can be attributed to systematic experimental error. Possible sources include incomplete thermal insulation of the Styrofoam calorimeter, which allows heat loss to the surroundings; slow response of the temperature probe, leading to underestimation of the true temperature change; or miscalibration of the calorimeter heat capacity, which propagates error through $q_{\text{rxn}}$ and $\Delta H_{\text{sol}}$. Each of these factors would result in a smaller observed temperature change and therefore a lower calculated enthalpy.

While both results describe KCl’s behavior in water, they suggest opposing effects on deicing performance. The van’t Hoff factor ($i = 1.90$) shows that KCl dissociates effectively and should lower the freezing point of water. However, its endothermic dissolution ($\Delta H_{\text{sol}} = 1.4\;\text{kJ·mol}^{-1}$) means it absorbs heat from the surroundings, making it less effective at melting ice compared to exothermic salts like CaCl$_2$. Thus, KCl can act as a deicer, but not an efficient one in very cold conditions.

\Needspace{3\baselineskip}
\section*{Conclusion}

The purpose of this experiment was to evaluate KCl as a deicer. The van’t Hoff factor and enthalpy of dissolution for potassium chloride were determined to evaluate its ionic behavior and thermodynamic suitability as a deicing agent. From the freezing‐point depression data, the van’t Hoff factor was found to be $i = 1.90$, indicating that KCl dissociates almost completely into $K^+$ and $Cl^-$ ions in aqueous solution. The measured enthalpy of dissolution was $\Delta H_{\text{sol}} = 1.4\;\text{kJ·mol}^{-1}$, confirming that the process is endothermic. These results align qualitatively with literature expectations for KCl, which dissolves endothermically and effectively lowers the freezing point of water through ionic dissociation.

Future work could extend this experiment by comparing the thermodynamic and colligative properties of KCl with other common deicing salts such as NaCl or CaCl$_2$, or by using differential scanning calorimetry to obtain higher‐precision enthalpy data. These extensions would clarify how ionic charge and lattice energy influence both dissolution energetics and freezing‐point depression.

On a broader scale, this work demonstrates why potassium chloride and related salts are effective for winter road treatment: their dissociation disrupts water’s ability to form ice. Understanding the thermodynamics behind this process helps improve deicer formulations that minimize environmental impact while maintaining safety for the public.

\newpage
\Needspace{3\baselineskip}

% References in ACS style
\begin{thebibliography}{9}

\bibitem{ref1} Federal Highway Administration. \textit{How Do Weather Events Impact Roads?} U.S. Department of Transportation, Office of Operations. \url{https://ops.fhwa.dot.gov/weather/weather\_events/snow\_ice.htm} (accessed Sept 29, 2025).

\bibitem{ref2} Kemsley, J. \textit{Why Does Salt Melt Ice?} \textit{ChemMatters}, American Chemical Society, February 2006. \url{https://www.acs.org/content/dam/acsorg/education/resources/highschool/chemmatters/articlesbytopic/solutions/chemmatters-feb2006-salting-roads.pdf} (accessed Sept 29, 2025).

\bibitem{ref3} Lide, D. R., Ed. \textit{CRC Handbook of Chemistry and Physics}, 85th ed.; CRC Press: Boca Raton, FL, 2004. \url{https://ia601308.us.archive.org/27/items/CRC.Press.Handbook.of.Chemistry.and.Physics.85th.ed.eBook-LRN/CRC.Press.Handbook.of.Chemistry.and.Physics.85th.ed.eBook-LRN.pdf} (accessed Sept 29, 2025).

\bibitem{ref4} Lide, D. R., Ed. \textit{CRC Handbook of Chemistry and Physics}, 85th ed.; CRC Press: Boca Raton, FL, 2004. \url{https://ia601308.us.archive.org/27/items/CRC.Press.Handbook.of.Chemistry.and.Physics.85th.ed.eBook-LRN/CRC.Press.Handbook.of.Chemistry.and.Physics.85th.ed.eBook-LRN.pdf} (accessed Sept 29, 2025).

\end{thebibliography}

\end{document}
