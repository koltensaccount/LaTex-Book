\documentclass[11pt,letterpaper]{article}
% Shared preamble for CHEM 1066 lab reports.
% This file loads the common styling and establishes course-wide defaults.
\usepackage{styles/chem1066-report}

% Set defaults that apply to every report; individual `.tex` files can override.
\SetChemReportCourse{CHEM 1066}
\SetChemReportInstitution{University of Minnesota -- Twin Cities}
\SetChemReportSemester{Fall 2025}
\SetChemReportAuthor{[Author]}
\SetChemReportHeader{Lab Report}
\SetChemReportPartners{[Lab Partners]}
\SetChemReportCoverTitle{Type Your Title Here:}
\SetChemReportCoverSubtitle{Concise and Informative Subtitle}
\SetChemReportPreparedOn{\today}

% Theme palette (choose exactly one)
\ChemReportUsePaletteDefault
% \ChemReportUsePaletteForestGold
% \ChemReportUsePaletteCoolTwilight


% Update these metadata values per report to keep headers and the cover page in sync.
\SetChemReportHeader{Lab Report \#}
\SetChemReportCoverTitle{Type Your Title Here:}
\SetChemReportCoverSubtitle{Concise and Informative Subtitle}
\SetChemReportPartners{[Lab Partners]}
\SetChemReportPreparedOn{\today}

\begin{document}

\MakeChemReportCoverPage

% ==============================
% Abstract
% ==============================
\begin{center}
\section*{\centering Abstract}
\begin{adjustwidth}{1.5cm}{1.5cm}
\textit{[150–250 words summarizing purpose, method, quantitative results (with units), and conclusion. No citations.]}
\end{adjustwidth}
\end{center}

% ==============================
% Introduction
% ==============================
\Needspace{3\baselineskip}
\section*{Introduction}
[Provide scientific context, relevant theory, and objectives. Use equations like $\Delta T_f = iK_f m$. Cite key references.\cite{ref1}]

% ==============================
% Experimental
% ==============================
\Needspace{3\baselineskip}
\section*{Experimental}
[Summarize procedure in paragraph form, past tense. Include chemicals, instruments, and conditions. Enough detail for reproducibility.]

% ==============================
% Results
% ==============================
\Needspace{3\baselineskip}
\section*{Results}
[Present data clearly with tables and figures. Units in headers. Include one sample calculation. Summarize results concisely.]

% --- Example Table ---
\begin{table}[h]
\centering
\caption{Measured freezing points of solutions}
\begin{tabular}{lcc}
\hline
Solution & Concentration (mol/kg) & Freezing Point (°C) \\
\hline
NaCl     & 0.50                   & -1.85 \\
NaCl     & 1.00                   & -3.72 \\
Urea     & 1.00                   & -1.86 \\
\hline
\end{tabular}
\end{table}

% --- Example TikZ Diagram ---
\begin{figure}[h]
\centering
\begin{tikzpicture}[scale=1]
\draw[thick,->] (0,0) -- (5,0) node[right] {x-axis};
\draw[thick,->] (0,0) -- (0,4) node[above] {y-axis};
\draw[thick] (0,0) -- (4,3) node[midway, above] {Sample line};
\end{tikzpicture}
\caption{Simple diagram drawn with TikZ.}
\end{figure}

% --- Example PGFPlot (Styled) ---
\begin{figure}[h]
\centering
\begin{tikzpicture}
\begin{axis}[
    width=0.75\textwidth,
    xmin=0, xmax=1.9,
    ymin=0, ymax=6.2,
    xlabel={Molality (mol/kg)},
    ylabel={$\Delta T_f$ (°C)},
    tick align=outside,
    grid=both,
    grid style={line width=.3pt, draw=gray!25},
    major grid style={line width=.5pt, draw=gray!50},
    minor grid style={line width=.2pt, draw=gray!15},
    legend style={draw=black, fill=white, font=\small, fill opacity=0.8},
    legend cell align={left},
    legend pos=south east
]
\addplot[
    only marks,
    mark=*,
    mark size=3pt,
    draw=\ChemReportSecondaryColor,
    fill=\ChemReportSecondaryColor!60!white
] coordinates {
    (0.5,1.85) (1.0,3.72) (1.5,5.60)
};
\addlegendentry{Data}
\addplot[thick, domain=0:1.9, draw=\ChemReportPrimaryColor] {3.53*x};
\addlegendentry{Best Linear Fit Line}
\end{axis}
\end{tikzpicture}
\caption{Freezing‐point depression versus molality with best‐fit line.}
\end{figure}

% ==============================
% Discussion
% ==============================
\Needspace{3\baselineskip}
\section*{Discussion}
[Interpret data. Compare with literature. Discuss sources of error. Connect back to theory and objectives.]

As shown in Figure~\ref{fig:titration-curve}, the pH rises sharply near the equivalence point, consistent with the expected titration behavior.

\begin{figure}[h]
\centering
\begin{tikzpicture}
\begin{axis}[
    width=0.75\textwidth,
    height=6cm,
    xmin=0, xmax=25,
    ymin=2, ymax=12,
    xlabel={Volume of Titrant (mL)},
    ylabel={pH},
    tick align=outside,
    grid=both,
    grid style={line width=.3pt, draw=gray!25},
    major grid style={line width=.5pt, draw=gray!50},
    minor grid style={line width=.2pt, draw=gray!15},
    grid style={line width=.3pt, draw=gray!25},
    major grid style={line width=.5pt, draw=gray!50},
    clip=false,
    legend style={at={(1.04,0.5)}, anchor=west, draw=black, fill=white, font=\small, fill opacity=0.9, align=left},
    legend cell align={left},
    samples=200,
    domain=0:25
]
% Sigmoidal titration curve with sharp transition near 12.5 mL
\addplot[thick, draw=\ChemReportPrimaryColor] (
    x,
    {2 + 10/(1 + exp(-0.154*(x - 12.5)))}
);
\addlegendentry{Titration Curve}

% Equivalence point vertical dashed line
\draw[thick, dashed, color=\ChemReportSecondaryColor] (axis cs:12.5,2) -- (axis cs:12.5,12);
\addlegendimage{thick, dashed, color=\ChemReportSecondaryColor}
\addlegendentry{Equivalence Point}

% Buffer region vertical dashed lines and shaded area
\draw[thick, dashed, color=\ChemReportPrimaryColor!50] (axis cs:2.5,2) -- (axis cs:2.5,12);
\draw[thick, dashed, color=\ChemReportPrimaryColor!50] (axis cs:10,2) -- (axis cs:10,12);
\addlegendimage{thick, dashed, color=\ChemReportPrimaryColor!50}
\addlegendentry{Buffer Region}
\addplot [ name path=bufferbottom, domain=2.5:10, draw=none, forget plot ] { 2 };
\addplot [ name path=buffertop, domain=2.5:10, draw=none, forget plot ] { 12 };
\addplot [ fill=\ChemReportPrimaryColor, fill opacity=0.15, forget plot ] fill between [ of=bufferbottom and buffertop ];

% pKa horizontal line, marker, and label
% 1. Plot the horizontal line
\addplot [ thick, dashed, color=\ChemReportAccentColor, domain=0:25 ] { 4.76 };
% 2. Add the legend entry for the line above
\addlegendentry { $ pK_a $ }

% 3. Plot the marker (dot) at half-equivalence
% We add 'forget plot' so it doesn't get its own legend entry
\addplot [ 
  only marks, 
  mark=*,
  mark size= 2.3pt ,
  draw= \ChemReportAccentColor ,
  fill= \ChemReportAccentColor,
  forget plot
] coordinates { (6.25,4.76) } ;

% Labels removed to avoid clutter
%\node[anchor=south west, color=\ChemReportSecondaryColor, font=\small] at (axis cs:12.5,12) {Equivalence Point};
%\node[anchor=south west, color=\ChemReportPrimaryColor!50, font=\small] at (axis cs:2.5,12) {Buffer Region};
%\node[anchor=south west, color=\ChemReportPrimaryColor!50, font=\small] at (axis cs:10,12) {Buffer Region};
\node[anchor=north east, color=\ChemReportAccentColor, font=\small] at (axis cs:25,4.76) {$pK_a$};
\end{axis}
\end{tikzpicture}
\caption{Example titration curve with key regions highlighted.}
\label{fig:titration-curve}
\end{figure}

% ==============================
% Conclusion
% ==============================
\Needspace{3\baselineskip}
\section*{Conclusion}
[Summarize findings concisely. Confirm whether objectives were achieved. No new data.]

% ==============================
% References
% ==============================
\newpage
\Needspace{3\baselineskip}
\begin{thebibliography}{9}
\bibitem{ref1} Author, A. B.; Author, C. D. \textit{Journal Name} \textbf{Year}, \textit{Volume}, page–page.  
\bibitem{ref2} Author, E. F. \textit{Book Title}; Publisher: Place, Year.  
\bibitem{ref3} Author, G. H. Title of Webpage. URL (accessed Sept 29, 2025).
\end{thebibliography}

\end{document}
