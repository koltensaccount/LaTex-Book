% Chapter 1 introduces the overall document shell and navigation helpers
\ChapterHeading{1}{Document Structure and Navigation}
\label{ch:document-structure}

\SectionBar{1.1}{Framing a Chapter}
\label{sec:chapter-frame}

\MarginFigureAuto[0pt]{%
  \begin{tikzpicture}[scale=0.72]
    \draw[gray!35,->] (-2,0) -- (2,0) node[right]{x};
    \draw[gray!35,->] (0,-1.5) -- (0,1.5) node[above]{y};
    \draw[Primary,thick,smooth] plot[domain=-2:2] (\x,0.35*\x);
    \draw[Highlight,thick,smooth] plot[domain=-2:2] (\x,0.1*\x*\x);
  \end{tikzpicture}
}{Basic line and curve showing the space reserved by \texttt{\textbackslash ChapterHeading}.}

Use \verb|\ChapterHeading| to stamp the title block and reset counters for
equations, theorems, examples, and margin figures. The macro also updates running
headers, so each chapter advertises itself in the page margin. A minimal chapter
file starts with
\begin{verbatim}
\ChapterHeading{<chapter number>}{<chapter title>}
\label{ch:...}
\end{verbatim}
Always follow the heading with a \verb|\label|. Later cross references, such as
Chapter~\ref{ch:document-structure}, rely on unique chapter labels.

\SectionBar{1.2}{Naming Sections and Subsections}
\label{sec:section-heads}

\MarginFigureAuto[4em]{%
  \begin{tikzpicture}[scale=0.8]
    \draw[gray!35,->] (-2.1,0) -- (2.1,0) node[right]{n};
    \draw[gray!35,->] (0,-1.4) -- (0,1.4) node[above]{s(n)};
    \draw[Primary,thick] (-2,1.1) -- (-1,0.55) -- (0,0) -- (1,-0.55) -- (2,-1.1);
    \foreach \x/\y in {-2/1.1,-1/0.55,0/0,1/-0.55,2/-1.1}{
      \fill[Highlight] (\x,\y) circle (1.4pt);
    }
  \end{tikzpicture}
}{Step graph showing how numbered headings drop into place.}

Section headers use \verb|\SectionBar{<number>}{<title>}|. The number argument is
flexible: supply either the exact printed identifier (e.g. \verb|1.2|) or a short
descriptor like \verb|Remark| for specialised sections. Pair each heading with a
label, such as Section~\ref{sec:section-heads}, so the table of contents can link
directly into the narrative.

The paragraph spacing after a section heading is tuned for single-paragraph
introductions. If you need more room before the first line, insert a manual
\verb|\vspace| adjustment beneath the heading—keep those changes local so the
default rhythm stays intact elsewhere.

\SectionBar{1.3}{Referencing Core Elements}
\label{sec:core-references}

Referencing is central to the style:

\begin{itemize}[leftmargin=1.8em]
  \item \verb|\ref{label}| prints the counter (Chapter~\ref{ch:document-structure}).
  \item \verb|\nameref{label}| prints the friendly name (\nameref{sec:section-heads}).
  \item \verb|\eqref{label}| encloses the equation number, as in
        Equation~\eqref{eq:sample-series}.
\end{itemize}

\begin{equation}
  \label{eq:sample-series}
  s(x) = \sum_{k=0}^{n} a_k x^k
\end{equation}

Give every float, theorem-style box, example, and equation a descriptive label so
readers mapping your notes to the examples here never guess which object you cite.

\SectionBar{1.4}{Where Definitions Enter}
\label{sec:definition-demo}

\begin{definitionBox}{Styled Definition}
\label{def:styled-definition}
A \emph{styled definition} pairs semantic emphasis with a consistent caption. The
environment \verb|\begin{definitionBox}{<title>} ...| automatically supplies the
definition counter and the accent panel.
\end{definitionBox}

\begin{theoremBox}{Spacing Checklist}
\label{thm:spacing-checklist}
Theorem boxes follow the same pattern. Keeping their counters in sync means you can
reference Theorem~\ref{thm:spacing-checklist} or Definition~\ref{def:styled-definition}
without additional macros.
\end{theoremBox}

\begin{textexample}{Quick Reference Card}
\label{ex:reference-card}
\[
  \chi(y) = y^2 - 1
\]
\solutiontag\; Use the example environment for printable study cards: its banner and
solution tag mimic the final design.
\end{textexample}

Examples and definitions are natural places to remind readers how to refer back to
earlier material. For instance, Example~\ref{ex:reference-card} cites
Definition~\ref{def:styled-definition} to show cross-links between environments.
