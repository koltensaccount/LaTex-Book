% Chapter 3 walks through theorem-style boxes, examples, and cross-referencing
\ChapterHeading{3}{Structured Statements and Examples}
\label{ch:structured-content}

\SectionBar{3.1}{Choosing the Right Box}
\label{sec:choose-box}

\MarginFigureAuto[0pt]{%
  \begin{tikzpicture}[scale=0.82]
    \draw[gray!35,->] (-2.1,0) -- (2.1,0) node[right]{x};
    \draw[gray!35,->] (0,-1.6) -- (0,1.6) node[above]{y};
    \draw[Primary,thick,smooth] plot[domain=-2:2] (\x,0.45*\x);
    \draw[Highlight,thick,smooth] plot[domain=-2:2] (\x,-0.45*\x);
  \end{tikzpicture}
}{Intersecting lines acting as a quick visual cue for Chapter~\ref{ch:structured-content}.}

Three pre-styled environments give your mathematical prose consistent rhythm:
\begin{description}[leftmargin=1.8em,labelwidth=2.8em]
  \item[\texttt{theoremBox}] For claims and proofs.
  \item[\texttt{definitionBox}] For vocabulary and notation.
  \item[\texttt{textexample}] For worked exercises with solutions.
\end{description}

Each increments a chapter-specific counter, ensuring Definition~\ref{def:sample}
and Theorem~\ref{thm:sample} share the same chapter index. Use descriptive titles
to orient readers before they dive into the details.

\SectionBar{3.2}{Demonstration Boxes}
\label{sec:demo-boxes}

\begin{definitionBox}{Sample Definition}
\label{def:sample}
Let \(V\) be a vector space. A \emph{styled span} is the set
\(\operatorname{span}\{v_1,\dots,v_k\}\) embellished with the accent palette defined in
\nameref{sec:choose-box}. Nothing about the mathematics changes—only the wrapper.
\end{definitionBox}

\begin{theoremBox}{Sample Theorem}
\label{thm:sample}
Styled spans form a subspace of \(V\). This mirrors traditional linear algebra
results while showcasing how theorem captions pick up the highlight colour.
\end{theoremBox}

\begin{textexample}{Referencing Styled Spans}
\label{ex:styled-span}
\[
  T(\mathbf{x}) = A\mathbf{x},\qquad A =
  \begin{pmatrix}
    1 & 2 \\
    0 & 1
  \end{pmatrix}
\]
\solutiontag\; Reference Definition~\ref{def:sample} when explaining why the column
space of \(A\) is a styled span. The example banner stretches into the margin to
create a visual entry point for students scanning for practice problems.
\end{textexample}

Pair these environments to build exposition, proposition, and practice sections
without writing new TikZ or colour definitions each time.

\SectionBar{3.3}{Layering Cross References}
\label{sec:layering-crossrefs}

When composing longer notes, tie statements together explicitly:
\begin{itemize}[leftmargin=1.8em]
  \item Mention Definition~\ref{def:sample} before calling on Theorem~\ref{thm:sample}.
  \item Cite Example~\ref{ex:styled-span} in homework instructions so readers know
        which worked solution to re-read.
  \item Mix chapter references—compare this chapter with the margin guidance in
        Chapter~\ref{ch:figures} to explain where visual aids should appear.
\end{itemize}

Explicit cross references keep the instructional tone coherent. They also make it
easy to search the compiled PDF for the label names when students ask follow-up
questions.

\SectionBar{3.4}{Quick Author Checklist}
\label{sec:author-checklist}

Before moving on to subject-matter writing, verify the following:
\begin{enumerate}[leftmargin=1.8em]
  \item Every theorem, definition, example, figure, and important equation has a label.
  \item Chapter introductions include a short paragraph that explains the chapter purpose.
  \item Margin figures sit next to their references (adjust offsets as needed).
  \item The table of contents lists only the sections you intend to publish.
\end{enumerate}

The checklist above doubles as a template for future reference chapters—copy the
format, change the text, and reuse labels with new chapter prefixes to keep the
structure predictable.
