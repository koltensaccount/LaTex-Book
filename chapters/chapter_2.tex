% Chapter 2 concentrates on figures, margin assets, and float placement
\ChapterHeading{2}{Figures and Margin Assets}
\label{ch:figures}

\SectionBar{2.1}{Margin Figures in Practice}
\label{sec:margin-figures}

\MarginFigureAuto[0pt]{%
  \begin{tikzpicture}[scale=0.8]
    \draw[gray!35,->] (-2,0) -- (2,0) node[right]{x};
    \draw[gray!35,->] (0,-1.5) -- (0,1.5) node[above]{y};
    \draw[Primary,thick,smooth,samples=80] plot[domain=-2:2] (\x,{0.6*sin(deg(\x))});
  \end{tikzpicture}
}{Every call to \texttt{\textbackslash MarginFigureAuto} produces a numbered figure in the outer margin.}

\verb|\MarginFigureAuto| accepts an optional vertical offset, the TikZ (or image)
content, and the caption text. Because the template enforces left-side placement,
margin art stays aligned with the spine no matter which page holds it. When you
reference Figure~\ref{fig:2.1}, readers can scan margins to locate the artwork
without worrying about parity.

Spacing tip: tweak the optional offset (default \verb|0pt|) to nudge a figure
toward the paragraph it supports. Positive values move the art downward; negative
values pull it upward.

\SectionBar{2.2}{Body Floats and Anchors}
\label{sec:body-floats}

\begin{figure}[H]
  \centering
  \begin{tikzpicture}[scale=1]
    \draw[gray!40,->] (-3,0) -- (3,0) node[right]{x};
    \draw[gray!40,->] (0,-1.8) -- (0,1.8) node[above]{y};
    \draw[Primary,thick,smooth,samples=80] plot[domain=-2.5:2.5] (\x,{0.6*sin(deg(\x))});
  \end{tikzpicture}
  \caption{A centred sine curve anchored with the \texttt{[H]} placement specifier.}
  \label{fig:body-float}
\end{figure}

For figures inside the main text column, combine the standard \verb|figure|
environment with the \verb|H| placement specifier from the \texttt{float} package
to anchor the illustration near its mention. Without \verb|[H]|, LaTeX may shuffle
the figure earlier or later to fill whitespace, which is why
Figure~\ref{fig:body-float} would otherwise drift toward the top of the chapter.

Keep captions brief and action-oriented. They sit directly under the illustration
and are automatically numbered for cross references like Figure~\ref{fig:body-float}.

\SectionBar{2.3}{Inline Callouts}
\label{sec:inline-callouts}

Margin figures work best when paired with in-text cues:
\begin{itemize}[leftmargin=1.8em]
  \item Signal the reader using ``see Figure~\ref{fig:2.1}'' language.
  \item Reserve \verb|\marginnote| for short callouts; longer commentary fits better
        in the main column.
  \item When multiple figures appear back-to-back, mention each explicitly so the
        reader knows which panel matches your explanation.
\end{itemize}

If you later replace a TikZ sketch with an imported image, swap the drawing code
with \verb|\includegraphics[width=\linewidth]{<file>}| inside
\verb|\MarginFigureAuto|. The surrounding macro keeps the caption styling identical.
